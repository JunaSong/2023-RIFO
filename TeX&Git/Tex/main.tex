\documentclass[a4paper,12pt]{article}
\usepackage{graphicx}
\usepackage{subcaption}
\usepackage{amsmath}
\usepackage[cjk]{kotex}
\usepackage{ulem}

\begin{document} % start
%제목
\begin{center} % 중앙 정렬
\LARGE {신입생 교육 2일차} \\
\vspace{10pt} % 줄 사이 간격 조절
\small {LaTex(레이텍스) 쓰기}
\end{center}

\begin{flushright} % 오른쪽 정렬
%날짜
2022.12.29

\end{flushright} % 한줄 간격 넣기
\begin{flushright}
%이름
\textit{Jungyeon Lee} % 이텔릭체
\rule{\linewidth}{1pt}
\end{flushright}

\begin{flushleft}
\section*{Last Week} % 섹션
\end{flushleft}

\begin{enumerate} % 번호 붙이기

\item 자기소개 % item 하나의 컨텐츠로 인식
%이름, 소속 팀, 취미, 주의사항
\begin{itemize} % 들여쓰기
    \item 이름: 
    \item 나이: 
    \item 소속: 
    \item 입학년도: 
    \item MBTI: 
    \item 간단한 소개: 
\end{itemize}

\item 연습
\begin{itemize}
	\item 수식 쓰기 : $x+y$
	\item \textbf{사진 올리기} % crtl+b 로 볼드 i 로 이텔릭
	\begin{figure}[h] % here
	    \centering
	    \includegraphics[scale=1.5]{logo.png} % 사진 크기 조정
	    \caption{Robotics Innovatory}
	    \label{fig:my_label}
	\end{figure}

\end{itemize}

\end{enumerate} % 번호 붙이기 끝

\newpage % 다음페이지에서 작성

\begin{flushleft}
\section*{Future Plan}
\end{flushleft}

\begin{enumerate}

\item 

\end{enumerate}
\end{document} % 끝